\documentclass[12pt]{article}


\usepackage[brazilian]{babel}
\usepackage[utf8x]{inputenc}
\usepackage{amsmath}
\usepackage{graphicx}
\usepackage[colorinlistoftodos]{todonotes}
\usepackage{geometry}
\usepackage{verbatim}
\usepackage{microtype}
\usepackage[numbers]{natbib}
\usepackage[hidelinks]{hyperref}
\usepackage{amsfonts}
\usepackage{color}
\geometry{a4paper}

\title{Um sistema Web para execução remota de aplicações de alto desempenho}
\author{Otávio Migliavacca Madalosso \\ \emph{Universidade Federal de Santa Maria}}

\begin{document}
\maketitle


\section{Identificação}

\begin{description} \itemsep 0pt

\item{\textbf{Resumo:}}Algumas áreas de pesquisa utilizam constantemente algoritmos que demandam alto desempenho dos seus ambientes de execução. Ocasionalmente, surgem algoritmos novos, com diferentes propriedades, que se propoem a resolver um problema de forma mais eficiente e/ou completa. Infelizmente, é comum que esses algoritmos fiquem restritos a ambientes institucionais, limitando muito a sua visibilidade para a comunidade de pesquisa. Este trabalho tem como objetivo criar um portal que permita ao usuário solicitar a execução remota de algoritmos de acordo com as configurações que ele desejar.


\item{\textbf{Período de Execução}}: Setembro de 2015 até Dezembro de 2015

\item{\textbf{Unidades Participantes}}: 

    Curso de Ciência da Computação
    
    Laboratório de Sistemas de Computação
    
    
\item{\textbf{Área do Conhecimento}}:Ciência da Computação

\item{\textbf{Linha de Pesquisa}}:?

\item{\textbf{Titulo do Projeto}}: Trabalho de Conclusão de Curso

\item{\textbf{Participantes}}:
\\Profª Andrea Schwertner Charão - Orientadora
\\Otávio Migliavacca Madalosso - Orientando
\\ 
\end{description}


\section{Introdução}

Algoritmos com grande custo computacional são facilmente encontrados em áreas como meteorologia, biologia e astronomia. Esses algoritmos possuem a característica de utilizar um nível elevado de processamento para concluir sua execução, e consequentemente, seus tempos de execução podem variar dependendo da máquina aonde estão sendo executados.

É comum pesquisadores destas e de outras áreas desenvolverem novas implementações de algoritmos utilizados por seus colegas. Implementações essas que podem trazer muitos benefícios para outros pesquisadores que necessitam deste tipo de solução. Infelizmente, é comum essas implementações ficarem restritas a ambientes privados, não por questões de licença, mas simplesmente pela ausência de um método prático para disponibilizar a nova ferramenta ao público.

Com isso, surge a ideia de desenvolver um portal web que permita ao usuário o cadastro de um experimento, no qual ele poderá ditar os dados de entrada desse experimento, e qual algoritmo (disponível no sistema) ele deseja utilizar para processar os dados. Depois de requisitar o experimento, o sistema deve providenciar sua execução e quando finalizar, retornar o resultado do experimento ao usuário que o requisitou.

Para este portal, será utilizado um algoritmo desenvolvido para a área de astronomia, uma versão do algoritmo Friends-of-Friends de complexidade n*log(n) paralelizada através do framework OpenMP

\section{Objetivos}
\subsection{Objetivo Geral}
O objetivo deste trabalho é criar um portal web que possibilite aos usuários cadastrados no sistema  executar algoritmos presentes no sistema segundo suas configurações e disponibilizar o resultado da execução após o término da mesma.

\subsection{Objetivos Específicos}
\begin{itemize}
	\item Estudo de frameworks web para ser utilizado no desenvolvimento.
	\item Desenvolvimento front-end do servidor.
	\item Administração das execuções requisitadas.
	\item Atualização dos estados das requisições no sistema.
    
\end{itemize}

\section{Justificativa}
O projeto é capaz de gerar benefícios significativos para a comunidade de pesquisa de diversas áreas, criando um ambiente que facilite a divulgação e teste de resultados de algoritmos alternativos para resolução de problemas comuns.

\section{Revisão de Literatura}
\section{Metodologia}
O método de pesquisa aplicada será utilizado, pois o projeto se propõe a aplicar os ensinamentos adquiridos no decorrer do curso para explorar as melhores soluções para atingir os objetivos.

\section{Plano de Atividades e Cronograma}
\begin{enumerate}
\item \label{activity:frameworks} \textbf{Estudo de Frameworks Web para desenvolvimento}
Será realizado um estudo dos frameworks mais conhecidos no mercado e selecionado o que mais apresentar benefícios para o desenvolvimento da aplicação proposta
\item \label{activity:develop} \textbf{Desenvolvimento front-end}
Após o estudo e seleção do framework para implementação, inicia-se o inicio do desenvolvimento do servidor front-end, criando o portal com as funções CRUD para gerenciar contas de usuários e os experimentos.
\item \label{activity:exec} \textbf{Administração das Execuções requisitadas}
Por se tratar de um portal que permitirá a execução remota de aplicações de alto desempenho, sabe-se que será necessário um estudo de como implementar a administração das requisições dos experimentos, visto que não deve existir concorrência de 2 ou mais experimentos no sistema.
\item  \label{activity:updates} \textbf{Atualização de estados dos experimentos}
Quando resolver-se a questão de execução dos algoritmos pelo sistema, também será necessário criar um método de verificar quando essas execuções finalizaram, bem como disponibilizar os resultados obtidos para o usuário no portal.
\end{enumerate}
\textbf{Cronograma}

\begin{table}[ht]
\centering
\begin{tabular}{c|ccccc}
	Etapa & Agosto & Setembro & Outubro & Novembro & Dezembro \\ \hline
	\ref{activity:frameworks} & \checkmark & & & \\
	\ref{activity:develop} & \checkmark & \checkmark & & \\
	\ref{activity:exec} & & \checkmark & \checkmark & \checkmark & \\
	\ref{activity:updates} & & &\checkmark & \checkmark & \checkmark \\
\end{tabular}
\caption{Cronograma de Atividades}

\end{table}
\section{Recursos}
Para a realização do projeto será utilizado o equipamento pessoal do orientando, nenhum \textit{hardware} específico será utilizado pois espera-se que seja desenvolvido um sistema que possa ser mantido em um servidor com configurações semelhantes com o de qualquer serivdor web.
\section{Resultados Esperados}
Ao término do trabalho, espera-se obter um sistema funcional compatível com o que foi explorado na sessão de Objetivos, um portal web que permita cadastro de usuários e no qual, cada usuário pode requisitar a execução de algoritmos existentes no sistema, utilizando para tal suas proprias configurações de execução (Dados de entrada, propriedades de execução).
\bibliographystyle{abbrvnat}
\bibliography{../graphics,../languages}

\end{document}